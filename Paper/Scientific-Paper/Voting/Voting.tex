\documentclass{article}
\usepackage{enumitem}
\usepackage[normalem]{ulem}
\usepackage{color}
\usepackage[hidelinks]{hyperref}
\usepackage{graphicx}
\usepackage[top=2cm,bottom=2cm,left=3cm,right=3cm]{geometry}
\usepackage{multicol}
\usepackage{float}
\usepackage{wrapfig}
\usepackage{csquotes}

%Used for bibliography
\usepackage[british]{babel}
\usepackage[%
  autolang=other,
  backend=bibtex      % biber or bibtex
%,style=authoryear    % Alphabeticalsch
 ,style=numeric-comp  % numerical-compressed
 ,sorting=none        % no sorting
 ,sortcites=true      % some other example options ...
 ,block=none
 ,indexing=false
 ,citereset=none
 ,isbn=true
 ,url=true
 ,doi=true            % prints doi
 ,natbib=true         % if you need natbib functions
]{biblatex}
\addbibresource{../Bibliography/Bibliography}



\begin{document}

    \section{Electronic Voting Systems}
    
    Existing electronic voting systems all suffer from a serious design flaw: They are centralized by design, meaning there is a single supplier that controls the code base, the database and the system outputs while also supplying the monitoring tools to verify the result \citep{3_noizat_2016}. The lack of an independently verifiable system means that, once voters mark their ballot choice, they must place their trust in the organization, that their vote is recorded and counted as intended. The lack of an independently verifiable output, makes it difficult for these centralized systems to acquire the trustworthiness required by voters, which potentially limits voter participation, or cast doubt upon the published output of an election.

Despite the digitalisation of many aspects of modern life, elections are still being largely conducted offline, on paper \citep{43_ernest_2014} although the use of Electronic Voting Machines has been steadily growing over recent years. Paper ballots are the traditional tool for voting and are typically marked by a human (voter) and then tallied by a machine. While costing less than most electronic systems to run they rely on physical security and trust in polling stations to not manipulate and to properly handle them \citep{44_wyndham_chen_das_2016}. Postal votes also utilize paper ballots and are used to allow voters to not have to physically attend a location in order to vote. These also suffer from the same flaws as traditional paper ballots while increasing the opportunities for attack during their traversal through the postage system.

Voting systems comprise of five main components: \verb|(1)| a registration service for verifying \& registering legitimate voters; \verb|(2)| voter authentication stations with the task of determining a voter’s authorization to vote based on the completed work of the registration service; \verb|(3)| voting stations where the voter makes choices on a ballot; \verb|(4)| a device called the ballot box where the ballot is collected; and \verb|(5)| a tallying service that counts the votes and announces the results. Of the five main components listed above, only \verb|(2)|, \verb|(3)| and \verb|(4)| are used during an election, with \verb|(1)| being being required for an election to take place and \verb|(5)| happening post election \citep{48_safevote_2001}.

Current Electronic Voting (E-Voting) takes two forms; using a machine in a polling station, rather than a ballot paper and pencil, or casting a vote over the internet. The former tends to refer to a Direct Recording Electronic system which typically displays ballot options on a screen that can be activated by the voter and then records that voting data in memory components to be processed later. However, as with many electronics there is an inherent problem of the ability to modify software and potentially insert malicious code \citep{44_wyndham_chen_das_2016}. This has been an issue raised over several recent elections and a study from 2015 concluded that 43 American states would be using Electronic Voting Machines that are at least 10 years old during the last presidential election \citep{45_holmes_2016}. The latter, while having to deal with issues such as privacy, fraud, voting under duress, and corruption, does nothing to improve a voter's trust. The election as the voter must assume that once they have cast their vote it will be recorded and counted honestly.

In order for our proposed E-Voting system to be a tangible challenger to more traditional voting methods it must be able to provide the current systems services, at least at the same level but preferably with improvements, while also providing substantial benefits to justify adoption. There are several standard requirements that a voting system should adhere to, each holding equal weight: security, functionality, privacy, usability, and accessibility \citep{46_voting_system_standards_testing_and_certification_2016}. A ``secure'' voting system means one that cannot be tampered with or manipulated in any way, ensuring that votes are accurately recorded as cast. It also ensures that additional votes cannot be cast after the polls have closed or tampered with at any stage of the process. System functionality can be broad but should include; The correct registering and recording of all votes cast, permitting a voter to vote for any candidate they have the right to vote for, allowing only eligible registered voters to vote \& only allowing each voter to vote once. Voters have the right to a secret ballot and to cast their vote in private \citep{46_voting_system_standards_testing_and_certification_2016}. This is essential to protect voters from being coerced or bribed into voting a certain way, this means that our system should not provide a receipt or any way for another person to determine the contents of a voter's ballot. On top of this the system should be easy for voters to use, meaning it's as intuitive as possible, and maintain universal vote access.  It should avoid introducing bias by selecting platforms that are more available to some groups than to others as the choice of the platform, language, ballot format, or devices may seem innocuous, but it may actually prevent small factions of the voters to cast their vote \citep{47_petride_2016}.

Whilst maintaining these essential foundations already provided in traditional voting systems, there are several improvements and additions which I intend to explore. The first benefit in using a blockchain to log votes is in its decentralized nature. This means there is less need for trust to be placed in a centralized organization where votes are hidden behind closed doors. It also has the benefit of being significantly harder to tamper with as, once a transaction has been verified, (as discussed previously) an attacker would need to possess at least 51\% of the computing power of the network to attempt to forge transactions. Any attempts to otherwise use a forged block will be noticed by the rest of the network and ignored. This decentralized system also brings in more transparency as anyone can view transactions in the blockchain leading to higher levels of trust in the elections outcome. This is further strengthened by the independent verifiability which could be performed by anyone, therefore removing the need to trust the election organizers declaration of the outcome. Once a transaction (vote) has been confirmed in the blockchain (and has further blocks built upon it) this vote for a candidate becomes immutable, meaning that the entire outcome of an election will be stored indefinitely and is able to be accessed at any time in the future.

There has been some research conducted in this area already and several protocols for blockchain based voting have been proposed. Pierre Noizat proposes a system \citep{3_noizat_2016} where each candidate provides a unique public key, KeyC, to each individual voter along with a singular bitcoin address, AddressC, which the final sum of the voting transactions will be sent to. Each voter is also assigned an individual public key, KeyA, by the election organizers and; either they generate a Key Pair themselves (this could be done by the voting software for better usability), KeyB, or be assigned the KeyB by the organization. From these three public keys the voter can generate a \verb|2-of-3| multi signature address which represents the vote of B in favor of C. This address is then funded with a bitcoin micropayment (around the price of a postage stamp), either funded by the voter or organization, and this is the voters confirmation of their vote. After a few hours this ballot is securely logged on the blockchain and, as the multisignature address was funded, a voter can check that this address is represented in the blockchain and that their vote was registered. It should be noted that there is no way to guess neither the voter (B) nor the candidate (C) from a multisignature address without knowing all three public keys (KeyA, KeyB, and KeyC) and knowing to whom they belong. Once the election is concluded the organization is able to, via the \verb|2-of-3| multi signature address, spend the coins the voter gave to the candidate to fund the address of the candidate, AddressC. This provides an unequivocal link between the vote and the candidate which can be seen and validated by anyone.

While the proposed method does provide a valuable alternative to current, proprietary electronic voting systems and has the benefits of protecting the secrecy of the ballots, allowing free, independent audits of the results and minimizing the trust level required from the organizers \citep{3_noizat_2016}. It does come with several drawbacks; the independent validation of votes before the organization funds a candidate's public address, AddressC, can only be done by the voting individual on their ballot choice. The protocols dependence on the Bitcoin blockchain could pose problems with subsequent elections as there is no definitive boundary between one election and the next. The currency units, although in very small denominations, could be transferred out of the election to private addresses for an individual's gain (though even if 100\% of the vote currency was lost, the cost of an election would likely be less than if done using current methods).

Another proposal is that of Universal Cast-as-Intended Verifiability \citep{49_escala_guasch_herranz_morillo_2015} which allows any party (not only the voter) to publicly verify that an encrypted cast vote really matches the selection of a voter’. Their proposal allows a voter who’s eligible to vote, to register with a registrar who then generates a pair of public-secret values for each voting option in the election. These secret values are sent to the voter, while the public ones are published, linked to the voting options they are related to. During the voting phase, the voter provides her selected voting options and a subset of the secret values she received during registration to the voting device. The voting device then encrypts the voter’s selections and creates a noninteractive zero-knowledge proof, which will be valid only in the case that the voting device encrypted what the voter selected. Thanks to the zero-knowledge property of the proofs, they can be publicly verified while maintaining the voter’s privacy.

While this may seem like a vastly superior proposal externally, the additional complexity of the underlying system, that is the inclusion of zero-knowledge proofs, should not be underestimated. Furthermore, zero-knowledge protocols, despite being proposed in the late 1970s, are still in their relative infancy when compared to Bitcoin (e.g. zCash whitepaper \citep{50_ben-sasson_chiesa_garman_green_miers_tromer_virza_2014}, 2014) and therefore have not been through the same level of scrutiny nor do they have the same level of development or adoption. The protocol also requires the voter to supply a secret value for each of the voting options they did not choose which, may require considerable effort if the ballot is large enough, and is counter intuitive to what would usually be expected.

\printbibliography
\end{document}