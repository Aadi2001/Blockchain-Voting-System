\documentclass{article}
\usepackage{graphicx}
\usepackage{standalone}
\usepackage{multicol}
\usepackage[parfill]{parskip}

% Used for sequence diagram
\usepackage{pgf-umlsd}
\usepackage{tikz}
\usetikzlibrary{shapes,arrows}

%used for subsubsubsections
\setcounter{tocdepth}{5}
\setcounter{secnumdepth}{5}

\usepackage{changepage}   % for the adjustwidth environment

% syntax highlighting
\usepackage{listings} 
\usepackage{color}
\definecolor{lightgray}{rgb}{.9,.9,.9}
\definecolor{darkgray}{rgb}{.4,.4,.4}
\definecolor{purple}{rgb}{0.65, 0.12, 0.82}
\definecolor{codegreen}{rgb}{0,0.6,0}
\definecolor{codegray}{rgb}{0.5,0.5,0.5}
\definecolor{codepurple}{rgb}{0.58,0,0.82}
\definecolor{backcolour}{rgb}{0.95,0.95,0.92}
\lstdefinelanguage{JavaScript}{
  keywords={typeof, new, true, false, catch, function, return, null, catch, switch, var, if, in, while, do, else, case, break},
  keywordstyle=\color{blue}\bfseries,
  ndkeywords={class, export, boolean, throw, implements, import, this},
  ndkeywordstyle=\color{darkgray}\bfseries,
  identifierstyle=\color{black},
  sensitive=false,
  comment=[l]{//},
  morecomment=[s]{/*}{*/},
  commentstyle=\color{purple}\ttfamily,
  stringstyle=\color{red}\ttfamily,
  morestring=[b]',
  morestring=[b]"
}

\lstset{
   language=JavaScript,
   backgroundcolor=\color{lightgray},
   extendedchars=true,
   basicstyle=\footnotesize\ttfamily,
   showstringspaces=false,
   showspaces=false,
   numbers=left,
   numberstyle=\footnotesize,
   numbersep=9pt,
   tabsize=2,
   breaklines=true,
   showtabs=false,
   captionpos=b
}

% Hyperlinks
\usepackage[colorlinks,allcolors=blue]{hyperref}
\usepackage[normalem]{ulem}
\usepackage{xcolor}
\makeatletter
\begingroup
  \catcode`\$=6 %
  \catcode`\#=12 %
  \gdef\href@split$1#$2#$3\\$4{%
    \hyper@@link{$1}{$2}{\uline{$4}}% or \underline
    \endgroup
  }%
\endgroup
%Used for bibliography
\usepackage[british]{babel}
\usepackage[%
  autolang=other,
  backend=bibtex      % biber or bibtex
%,style=authoryear    % Alphabeticalsch
 ,style=numeric-comp  % numerical-compressed
 ,sorting=none        % no sorting
 ,sortcites=true      % some other example options ...
 ,block=none
 ,indexing=false
 ,citereset=none
 ,isbn=true
 ,url=true
 ,doi=true            % prints doi
 ,natbib=true         % if you need natbib functions
]{biblatex}
\addbibresource{../Bibliography/Bibliography}

\begin{document}
    \section{Conclusion}
    
    \subsection{Overview}
Although this system has been designed and developed with the idea of a national general election in mind, the protocols and ideas involved could be applied to smaller scale ballots which wish to provide transparency in their audit. Although we wish to minimize trust in a central authority, due to the nature of these type of elections (where there needs to be some degree of voter eligibility verification), we cannot fully decentralize this system as we need to only allow those eligible the rights to vote. Despite needing to verify an individual we still need to ensure that their votes are publicly anonymous, especially given the public transactions underpinning the blockchain concept while providing the ability for an individual to verify that their vote was correctly counted.

I do not see this system as a direct ``replace all'' for national election voting. I believe there will still be a need for traditional voting implementations in certain situations; for example, maintaining postal vote for the elderly who may not have the technical capability or equipment for online voting. However I do think that this could be phased in along side traditional voting, eventually replacing the pre-existing e-voting systems and ultimately becoming the main way for the majority of people to choose their government. 

The designed schema for this protocol is the following:


\end{document}
